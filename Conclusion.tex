\chapter{Conclusion}
In this dissertation, we have proposed a computational framework that combine the Eulerian 
description of the turbulence field with a Lagrangian particles ensemble. 
This framework has been applied to the numerical study of parachute deceleration and 
cloud microphysics. Although these two fields have different backgrounds, they can be uniformly  
modeled with PDEs and ODEs, and numerically solved in the same framework.

The turbulence field is described by the incompressible Navier-Stokes equation, which uniquely 
determines the velocity and pressure field in the domain. However, if the computational grid is not 
fine enough to cover the entire length scale, directly solving the Navier-Stokes may fail 
to fully resolve the effects of turbulence kinetic-energy cascade and backscatter. In a statistically-averaged sense, the Naiver-Stokes equation can be modified using the Reynolds decomposition. The resulting velocity and pressure field are solved from the Reynolds-Averaged Navier-Stokes equation in the perspective of ensemble average, and the additional term of Reynolds stress can be approximated by the turbulence model. 

The Naiver-Stokes equation is essentially a special case of the nonlinear PDE, and its solving procedure contains the numerical schemes for hyperbolic, parabolic and elliptic equations. Therefore, these methods can be directly reused to solve the associated scalar field, which is described by a parabolic equation with a source or sink term. However, the appearance of the sink terms adds more challenges, since many physics problem requires the underlying field to be positive, such as kinetic energy, dissipation rate, temperature and concentration of chemical species. One way to overcome this difficulty is to linearize the sink term and set a lower bound for the coefficients without touching the final solution. This method keeps the positivity of the solution and not introducing any direct artifacts on it.

A particle system has been introduced in the framework due to its simple structure and wide range of application. There exist two kinds of particle system: independent or connected system. In the independent system, such as cloud droplets, the motion of one particle has no direct impacts on the rests, and hence the system can be decoupled to a set of independent equations. In the opposite case, such as the spring-mass model, the connected particle system consists of many coupled ODEs, which should be solved simultaneously. Moreover, the connected-particle system usually forms an interface, and hence an efficient collision treatment has been developed to avoid the surface self-intersections. In addition, the ODEs for the particle system can be stiff, thus a numerical method 
with large stability region is desirable. We have examined the implicit Euler method, explicit Runge-Kutta method and BDF method. We choose the appropriate method by considering its efficiency, 
accuracy, and stability.   

This framework has been used to study two applications: the parachute deceleration and cloud microphysics. The parachute simulation is carried in a turbulence environment with a relatively large Reynolds number, and thus the turbulence model is required. We have examined three different turbulence model: standard, RNG and Realizable $k$-$\epsilon$ model. The numerical experiment shows that the standard and RNG model tends to overestimate the turbulence effects when the flow is laminar while the Realizable model gives a reasonable prediction. In the meanwhile, we have proposed a new porosity model to simulate the porous effects of the parachute canopy. The new model is formulated by combining the Navier-Stokes equation with the Darcy's law through the Ghost Fluid scheme. This model is validated by the numerical experiments and concludes that the porosity effects is able to reduce the oscillation of the drag force. In addition, an efficient handling library has been developed to eliminate the intersections among the parachute canopy, suspension lines and cargoes. The collision treatment follows an iterative methodology, but is guaranteed to finish in finite number of steps thanks to the fail-safe method. The collision handling method is also applied to the folding of the parachute. We have achieved various folding patterns by combining some atomic operations.

The cloud entrainment and mixing process is studied in the same framework. Since the domain size is relatively small, we use direct numerical simulation to solve the turbulence field. The temperature field and vapor mixing ratio field are considered at the same time. The cloud droplets are described by the particle model, and can grow or shrink according to the local humidity. We have proposed three ways of configuration to model the cloud entrainment and mixing process at different location of the cloud. All the cases have been performed in both the decaying turbulence and forced turbulence. The main purpose of this study is to quantify the mixing degree, which is usually between the two extreme cases: the homogeneous mixing and inhomogeneous mixing. The numerical result suggests a new way to parameterize the mixing degree using the dynamical measures. Finally, the preferential concentration is also studied in this dissertation. The clustering index is used to quantify the degree of clustering, and we have examined the effects of the gravity on the clustering. The results imply that the gravity reduce the clustering effects in both the decaying and forced turbulence. Furthermore, a negative relationship between the number concentration and vorticity field has been verified through the experiment. It also shows that the gravity has less impacts on the forced turbulence than the decaying turbulence.
 
The two applications demonstrate that this computational framework is flexible and has a broad application. The PDE can be used to describe a wide variety of phenomena, such as sound, heat and fluid dynamics. These distinct physical problems can be formalized similarly in terms of PDE. In the meanwhile, the particle model is able to simulate various objects, from independent points to elastic membranes and rigid bodies, and hence adds more flexibility to this physically based modeling. The combination of PDE and particle system can be easily extended to study many other fields in the future.