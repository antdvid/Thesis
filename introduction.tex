\chapter{Introduction}
One of main goals in computational mathematics is to predict unknown situation or 
to study known natural phenomena via numerical simulation. This is achieved by employing 
mathematics equations to describe the physics phenomena, numerically solving these 
equations and analyzing the results to find the answer to the original question. 
Because not all the equations have an analytic solution and some phenomena may be too 
difficult to duplicate or observe through laboratory experiment, the numerical approach 
plays as an essential role in scientific research.

\section{Motivation}
In this dissertation, two research topics are proposed from atmospheric science and 
engineering field, and studied using numerical simulation and mathematics tool.
It will be shown later that even if these two topics are loosely related to each other 
as they appear, they can be described using a similar mathematics model and solved by
employing the same numerical method.

The first research studies the entrainment and mixing process at the cloudy-clear air interface. 
During the entrainment and mixing process, a preexisting cloud will capture and engulf a wind flow 
of entrainmental air deep into the cloud core. As turbulent mixing develops, these volumes are 
stretched and compressed into thinner filaments until the Kolmogorov scale is reached, where final 
homogenization occurs through the molecular diffusion process. Finally, the environmental air becomes 
part of the current cloud and eventually influence the entire cloud dynamics. Through condensation, 
evaporation, collision and coalescence, the entrainment and mixing process will change the distribution 
of water droplets in the cloud, which directly determine the cloud life time and rain formation. 
However, it was poorly understood that whether the reduction of liquid water in cloud 
was through the reduction of only droplet size, or only the number of droplets, or both the number 
and size. To quantify these phenomena helps capturing the details of turbulent transportation and 
dilution of cloud water, and provides significant parameters for representing clouds in coarse-resolution 
cloud model.

The second research focuses on the study of parachute deceleration system in turbulence flows.
Since building a qualified wind tunnel could be very expensive and testing a novel parachute in 
reality is extremely risky to the parachutist, numerical simulation is regarded as a safe 
and economical solution comparing to the laboratory experiment. It has been known that a lot of factors
may affect the inflation of the parachute, and therefore it is significant to know that in what condition
the parachute will fail to perform appropriately. For example, an angled drop of the parachute may lead
to the inversion of the canopy; the permeability of the parachute canopy can affect the stability of the 
deceleration; the parachute canopies in the parachute cluster may collide and twist with each other, and thus 
damage the deceleration process. Therefore, a deep understanding of these phenomena can help the engineer to 
improve the performance of the parachute as well as preventing some potential risks in reality.

Although numerical approach has a great advantage over the traditional experiment, there still exists 
several downsides and challenges for the numerical approach. Firstly, the correctness of the numerical method almost 
relies on the mathematics model, which may be not good enough to represent every detail of 
the original problem. Neglecting any features could differ the numerical results from the laboratory experiments. 
Secondly, the convergence and stability of the numerical algorithm
is still the main concern for the computational method. In spite of many progresses in the near decades, 
there is no such a "black-box" numerical algorithm that can solve any problem. For example, an effective 
method may become ineffective when applying to other problems; a method may fail if the environment changes 
significantly during the simulation; a combination of multiple high accurate 
methods may not achieve the desired order if they are not used appropriately.
Finally, some approach, like direct numerical simulation, requires a certain resolution to be effective.
This brings very high computational cost, sometimes exceed the capacity of the most powerful computers currently 
available.

The main goal of this dissertation is to develop an accurate and efficient numerical method to help understand 
the above problems. 
\section{Contribution}
The contributions presented in this dissertation have been published in several journals 
and conference proceedings. Below is the list of publications:

\section{Outline}
This dissertation is structured as follows.

\chapter{Literature review}
It's worthwhile to notice that these loosely related topics can be described 
using a similar model and numerically solved without any modifications. 
In the first study, the water droplets are represented as a group of particles interacting 
with turbulence environment. The particles are driven by the turbulence flow and has phase 
transition between water vapor and liquid. In the second study, the parachute canopy is built 
with a connected particle system. This system can be regarded as an elastic surface immersed in 
the surrounding fluid.

Particles are objects with no extent, while having mass $m$, position
$\mathbf{x}$ and velocity $\mathbf{v}$, and responding to forces $\mathbf{f}$.
Because they are simple, particles can be regarded as the easiest objects to
simulate. In spite of simplicity, particles can be powerful to model a
wide range of interesting behavior. For example, an unconnected particle group
can be used to simulate the water droplets in the research of cloud
microphysics; an elastic membrance can be built by connecting particles with
simple damped springs. 
\section{Particle model in cloth simulation}
 

\section{Particle model in entrainment and mixing}
