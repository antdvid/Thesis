\chapter{Introduction}
One of the main goals in computational mathematics is to predict unknown situation or 
to study known natural phenomena via numerical simulation. This is achieved by employing 
mathematics equations to describe the physics phenomena, numerically solving these 
equations and analyzing the results to find the answer to the original question. 
Because not all the equations have analytic solutions and some phenomena may be too 
difficult to duplicate or observe through laboratory experiment, the numerical approach 
plays as an essential role in scientific research.

\section{Motivation}
In this dissertation, two research topics are proposed from atmospheric science and 
engineering field respectively, and then studied using numerical simulation. 
It will be shown later that even if these two topics are loosely related to each other 
as they appear, they can be described using a similar numerical model and solved by
employing the same numerical method.

The first research studies the entrainment and mixing process at the cloudy-clear air interface. 
During the entrainment and mixing process, a preexisting cloud will capture and engulf a wind flow 
of environmental air deep into the cloud core. As turbulent mixing develops, these volumes are 
stretched and compressed into thinner filaments until the Kolmogorov scale is reached, where final 
homogenization occurs through the molecular diffusion process. Finally, the environmental air becomes a 
part of the current cloud and eventually influence the entire cloud dynamics. Through condensation, 
evaporation, collision and coalescence, the entrainment and mixing process will change the distribution 
of water droplets in the cloud, which directly determine the cloud life time and rain formation. 
However, it was poorly understood that whether the reduction of liquid water in cloud 
was through the reduction of only droplet size, or only the number of droplets, or both the number 
and size. To quantify these phenomena helps capturing the details of turbulent transportation and 
dilution of cloud water, and provides significant parameters for representing clouds in coarse-resolution 
cloud model.

The second research focuses on the study of parachute deceleration system in turbulence flows.
Since building a qualified wind tunnel could be very expensive and testing a novel parachute in 
reality is extremely risky to the parachutist, numerical simulation is regarded as a safe 
and economical solution comparing to the laboratory experiment. It has been known that a lot of factors
may affect the inflation of the parachute, and therefore it is significant to know that in what condition
the parachute will fail to perform appropriately. For example, an angled drop of the parachute may lead
to the inversion of the canopy; the permeability of the parachute canopy can affect the stability of the 
deceleration; the parachute canopies in the parachute cluster may collide and twist with each other, and thus 
damage the deceleration process. Therefore, a deep understanding of these phenomena can help the engineer to 
improve the performance of the parachute as well as preventing some potential risks in reality.

Although numerical approach has a great advantage over the traditional experiment, there still exists 
several downsides and challenges for it. Firstly, the correctness of the numerical method almost 
relies on the mathematics model, which may be not good enough to represent every detail of 
the original problem. Neglecting some features could differ the numerical results 
from the laboratory experiments. 
Secondly, the convergence and stability of the numerical algorithm
is still the main concern for the computational method. In spite of many progresses in numerical algorithm 
for differential equations, 
there is no such a "black-box" numerical algorithm that can solve any problem. For example, an effective 
method may become ineffective when applying to other problems; a method may fail if the environment changes 
significantly during the simulation; a combination of multiple high accurate 
methods may not achieve the desired order if they are not combined appropriately.
Finally, some approach, like direct numerical simulation, requires a certain resolution to be effective.
This brings very high computational cost, sometimes exceed the capacity of the most powerful computers currently 
available.

In consideration of the above problems, the main goal of this dissertation is to develop an realistic, robust, accurate 
and efficient numerical method, and apply this method to the study of entrainment mixing process in atmospheric science and parachute deceleration system in engineering field. In order to approach the reality, the related literatures are reviewed in order to choose the most suitable model to describe the physics. 
Some modifications are also needed to 
be desirable for our problems. Benefiting from the advantages of the modern numerical theory, the accuracy and robustness of the numerical has been improved and can be applied for general purpose. As for the computational cost, the parallel computing libraries, such as MPI and CUDA, are used to accelerate the computational speed and improve the efficiency.
It is true that not all the problems have been solved perfectly, but the experience and method developed in this dissertation can be applied to other problems related to ours.

\section{Contribution}
The projects in this dissertation are developed based on the \FronTierp 
framework \cite{}, which is  
originally designed for the study of fluid interface instabilities, but later has been 
used in many other applications. In order to solve the new problems using \FronTierp, many 
new features have been developed in this dissertation.   

A new second order finite difference method is developed to solve the convection 
diffusion equation for scalar field. This method is coupled with 
WENO (weighted essentially non-oscillatory) scheme, which has the capability of 
dealing with discontinuity without causing numerical overshoots at sharp interfaces.

The original \FronTierp library has the capability of dealing with complex geometry, such 
as 3d surfaces or 2d curves. However, in order to simulate the mass point 
in parachute canopy or water droplets in cloud, a new particle structure is added to 
\FronTierp. This new data structure allows us to decide the behavior of each individual particle without maintaining the underlying interface. The information of each particle, such as position, velocity, acceleration, are restored for later analysis.

A new porosity model is developed for the parachute canopy to simulate the average effects 
of the airflow through the permeable surface. This method is simple and effective since it 
successfully duplicates the porosity effect without adding too much computational cost. It 
also increases the robustness of the original method by adding a pressure at the interface 
position without smearing the symmetry of the coefficient matrix.

The original parachute code can only be used to simulate parachute behavior in laminar flow due  to its lack of turbulence model. To extend our work to the parachute motion in turbulence flow, a two equation RANS (Reynolds Averaged Navier Stokes) model is designed, implemented and coupled with the previous fluid solver. The new algorithm better predicts the turbulence transportation, and thus improve the accuracy of the parachute simulation.

One essential module for accurate simulation of the particle system dynamics is the collision handler, since two particles are not allowed to be too close to each other and the underlying interface should never interfere with itself. A new collision solver has been implemented to improve the accuracy of parachute simulation and well treat the self-interference during inflation. The new method makes use of the AABB (axis aligned bounding box) tree and union-find algorithm to improve the efficiency. In addition, the multi-threading computing technique OpenMP is used to increase the speed of traversing the particle group.       

The contributions presented in this dissertation have been published in several journals 
and conference proceedings. Below is the list of publications:

\begin{itemize}
\item Qiangqiang Shi, Daniel Reasor, Zheng Gao, Xiaolin Li, Richard D. Charles, ``On the verification and validation of a spring fabric for modeling 
parachute inflation", Journal of Fluids and Structures, 58, 2015
\item Zheng Gao, Richard D. Charles, Xiaolin Li, ``Numerical Modeling of flow through porous fabric surface in parachute simulation", AIAA Journal, 2016
\item Zheng Gao, Qiangqiang Shi, Yiyang Yang, Xiaolin Li, ``On verification and validation of spring fabric model", APS April Meeting, Baltimore, 2015.
\item Zheng Gao, Xiaolin Li, ``Numerical modeling and simulation of flow through porous fabric surface", APS March Meeting, 
Baltemore, 2016.
\item Zheng Gao, Yangang Liu, Xiaolin Li, ``Direct numerical simulation model for studying entrainment-mixing processes at 
sub-meter scales", Young Researcher Symposium, Brookhaven National Laboratory, 2015.
\item Yanggang Liu, Zheng Gao, Xiaolin Li, ``Studying Effects of Cloud Area Structure on Entrainment-Mixing Processes, Droplet Clustering, and Microphysics with a New Particle-Resolved Direct Numerical Simulation Model", AGU Fall Meeting, San Fransisco, 2016
\end{itemize}

\section{Outline}
This dissertation is structured as follows. 
Chapter \ref{literature} gives an extended overview of the literatures in 
parachute simulation, DNS of entrainment-mixing processes. 
In Chapter \ref{entrainment}, we provide a detailed study of the entrainment-mixing process, 
including the mathematics model, numerical implementation and discussion on the results.
Chapter \ref{parachute} describes the previous studies of the parachute simulation as well as several new features, such as turbulence, porosity, collision. The new model is validated by performing a few benchmark tests. Chapter \ref{conclusion} is our conclusion chapter.

\chapter{Literature Review}\label{literature}
In this chapter, an extended overview of the literature related to our research is 
given. Firstly, the previous numerical simulation of parachute is reviewed considering
various aspects, including general framework, cloth modeling, fluid dynamics, 
porosity modeling, collision handling and parallel computing. 
In the latter section, the past studies of entrainment and mixing process are discussed from both physics and numerical point of view. We also compare our method with the previous ones by discussing the advantages and disadvantages. 
The similarities and difference of the numerical method between parachute simulation and cloud simulation are also discussed.

\section{Parachute Simulation}
The parachute system is a complex system requiring
detailed study of many aspects in elastic mechanics and fluid dynamics. In the
past decades, many authors have made efforts to achieve an efficient
computational method for a realistic parachute simulation.  

In Takizawa and Tezduyar et al.'s series work \cite{Stein2000,Kalro2000}, 
a parallel computing platform based on finite element method was proposed for the 
parachute fluid-structure interactions. Their strategy is composed with three 
components: the fluid dynamics solution, structural dynamics solution, and the 
coupling of the fluid dynamics and structural dynamics along the fluid-structure 
interface. The fluid dynamics are modeled by RANS (Reynolds Averaged Navier-Stokes) equation coupled with Smagorinsky turbulence model, and then numerically solved with a stabilized space-time finite element method. The structural dynamics for the parachute system, such as cables and membranes, are solved with finite element formulation derived from the principle of virtual work. The fluid-structure interactions are
handled by an automatic mesh moving scheme, and transfer of necessary information between fluid dynamics mesh and structural dynamics nodes. Their works are then extended to a few types of parachute, such as parachute clusters \cite{Takizawa2010}, ringsail parachute \cite{Tezduyar2008}, cross parachute \cite{Stein2001} and spacecraft parachute \cite{Takizawa2013}.

Peskin and Kim developed a computational method based on the (IB) immersed boundary 
method for parachute simulation in \cite{Kim2006,Kim2009}. In their method, the
parachute was treated as an extremely thin shell immersed in the fluid field. Since no need for regeneration of mesh grid every time step, their algorithm was proved to be
more efficient than the finite element method in the expense of accuracy near
the interface. The original immersed boundary method is modified by adding penalty 
force in order to give mass to the parachute canopy, suspension lines, risers and payload. Their method has the advantages of handling large deformation of parachute canopy without re-meshing, as well as handling collisions without additional collision detection algorithm. It is also noticed that since lack of a realistic turbulence mode, the practical use of their works is limited to a very low Reynolds number. 

Tutt et al. utilized the transient dynamics finite element program LS-DYNA 
for the analysis of fabric structures and parachute system.
In \cite{Tutt2005}, an Eulerian-Lagrangian penalty
algorithm was used to replicate the inflation of round canopies in a water
tunnel. In \cite{Aquelet2007, tutt2006application, wang2006porous}, a new algorithm evaluating the fabric permeability is proposed based on the previous methods. The porosity method is simple and effective by considering the average aerodynamic motions of the porous canopy surface, instead of the fabric's porous
structure. The porosity effect is converted to the relationship 
between pressure drop and permeable velocity along the parachute interface based on the Ergun's law. This relationship is then incorporated into the fluid solver and 
affect the velocity and pressure field near the parachute surface.

In \cite{Kim2013,Li2013}, a mesoscale model
in effects to mimic the dynamic motion of a fabric surface was proposed and
coupled with an incompressible fluid solver for the study of parachute
inflating and descending. This method was then enhanced in \cite{Shi2015} by
incorporating the angular stiffness to the spring mass system, so as to
duplicate the physical properties of the continuum material, such as Young's
modulus and Poisson ratio.

In spite of many progresses in the simulation of parachute, to capture the
details of turbulent motions and its interactions with elastic material is
still a challenge in the computing field.  Firstly, the major concern on the
fabric simulation is the accurate modeling and prediction of the highly
non-linear behavior of cloth with the consideration of efficiency, stability,
and visual realism \cite{Choi2005}. Until now, the most successful physics
based simulation techniques were the interacting particle model proposed in the
computer graphics community, such as \cite{Baraff1998, Choi2002}. However, due
to the nonlinear hysteric properties of the fabric, the current simple linear
model will generate more characteristic of a rubbery material than of a cloth.
Simply introducing a prescribed non-linear law can still hardly duplicate the
real fabric behavior, but may increase the computational cost. The alternative
approach, continuum model with finite element method, has the advantage of
handling the nonlinearity, but still has problems in several factors, such as
high computational cost and non-trivial collision resolution.  

Secondly, the flow field around the parachute is usually high with Reynolds number of
several millions, which results in a development of the turbulent boundary
layer. This implies that an accurate and reliable turbulence model is
necessary for the calculation of drag force to the parachute canopy. However,
the lack of any comprehensive experimental data for the flow around parachute
makes an empirically justified choice extremely difficult. In addition, the
action of the flow in direct vicinity to the fabric surface is much stronger
than the rest of the boundary layer, thus its effect has to be taken into
account. Correct simulation of the flow in this viscous sublayer needs either
using the empirical wall laws or adjusting the turbulence model down to
the wall. All these factors may add some uncertainties to the parachute
simulation.  

Finally, besides elasticity, the most important property of the
parachute fabric is the permeability, which allows a fraction of the
free-stream flow go through the canopy surface and stabilizes the descent of
the parachute \cite{Johari2005}. An accurate simulation of the porous structure
in the pore level is extremely difficult with the current computational
resource. Therefore, proposing a relatively accurate but inexpensive
computational algorithm for porosity modeling becomes significant.

In this dissertation, we make some efforts to improve the previous methods and 
give some attempts to solve the problems proposed above. 
A computational study of parachute inflation is presented
through the application of the front tracking method, which is originally
designed for problem with moving interfaces \cite{Glimm1998, Tryggvason2001,
Hu2015}. The front tracking method has been implemented as an open source
software library named \textit{FronTier}. In this frame work, the fabric
surface is treated as a connected particle group whose motion is driven by
gravity, fluid pressure and interior force computed by spring mass model. The
surrounding airflow is described by the incompressible Navier-Stokes equation
and numerically solved by the projection method.  Besides the previous work,
we have made many other efforts to increase the realistically of the
simulation. Firstly, a simplified porosity model based on ghost fluid method
(GFM) \cite{Kang2000} is developed to model the fabric porosity. The main idea
of this algorithm is to add a resistance to the fluid field. This resistance is
then coupled with the projection method by affecting the pressure Poisson
equation. This method has the advantage of being consistent with Darcy's law
and keeping the symmetry of the coefficient matrix. Secondly, a robust fabric
collision handling algorithm based on \cite{Bridson2005} is implemented to
model the multi-parachute collision. This method applies the iteration
methodology and can guarantee no intersections in the fabric surface after a
successful handling procedure. Moreover, a rigid body is attached in place of
the load node, in order to consider the influence of the wake flow on the
parachute inflation. By varying the shape of the cargo, such as cube, sphere
and human body, different wake patterns can be achieved. Finally, a hybrid
computing technique, that combines the GPU computing of particles and
distributed parallel computing of fluid field, is used to accelerate the
computational speed.

\section{Cloud Microphysics and Entrainment-mixing processes}
The microphysics of cloud droplets in small-scale add some uncertainties to the
Large-eddy simulation (LES) models, by impacting on the spectrum of cloud
droplets during entrainment and mixing process  \cite{Jarecka2013}. It was
poorly understood that whether the reduction of liquid water content was
through the reduction of only the droplet size (extremely homogeneous mixing),
or only the number of droplets (extremely inhomogeneous mixing), or both the
number and size (inhomogeneous mixing). These situations are defined by
comparing two time scales: the turbulent homogenization time scale and
thermodynamic time scale associated with the evaporation process. In the
extreme cases, one time scale will dominate the other while in the unextreme
case the two scales are comparable. Recently, it has been demonstrated that
different mixing scenarios can occur and change during one single cloud
evolution (\cite{And09,Burnet07,Lehmann09}) and therefore it becomes more and
more important to have a reasonable and accurate estimation of the mixing
scenario for a sub-grid model. \cite{Jarecka2013}, utilized the DNS results
from \cite{And04,And06,And09}  to estimate the mixing scenario with a parameter
$\alpha$ and merged the $\lambda$-$\beta$ subgrid scheme (\cite{Jarecka2009})
with the double-moment LES model(\cite{Morrison2008}). \cite{Lu2013}, proposed
the transition scale number to measure the occurrence probability of
homogeneous or inhomogeneous entrainment-mixing process. Both of the cloud
observations and numerical simulations imply a positive relationship between
the transition scale number and the homogeneous mixing degree. In spite of many
progresses in the study of cloud microphysics, to capture the details of
turbulent transportation and dilution of cloud water is still a challenge for
representing clouds in coarse-resolution climate model.

As a supplementary to laboratory measurements (e.g., \cite{Malinowski2008}),
DNS allows extracting information that is difficult to obtain in the
laboratory. In specific, DNS gives a complete view of the mixing process at
various length scales from energy containing scale to the dissipation length
scale. It is also efficient to study the correlation between two different
variables by changing the parameters of the simulator. In the past decades,
many authors have contributed to the DNS of cloud microphysics through
continuum and discrete microphysics perspective. In Andrejczuk and Grabowski's
series work \cite{And04,And06,And09}, an incompressible Boussinesq
approximation model, along with vapor mixing ratio and temperature, is proposed
to study the cloud-clear air interfacial mixing. Different initial values of
turbulence kinetic energy (TKE) are used as input parameters to examine the
impact of turbulence intensity on droplet size distribution and mixing
homogeneity. The cloud filaments and velocity field are preset to make a
heavily idealized initial and boundary condition, while focusing on the details
of the evolving flow. \cite{Malinowski2008} compared the results of
\cite{And04,And06} with laboratory measurements. \cite{Lozar2014} added more
features such as sedimentation and particle inertial in the bulk formulation.
Other researchers studied the cloud microphysics by treating the droplet field
as discrete particles and explicitly tracking these particles. In
\cite{Lanotte2009} and \cite{Celani05}, a model combining Eulerian description
of the turbulent velocity and supersaturation fields with Lagrangian cloud
droplet ensemble was used to study the droplets condensation in turbulent
flows. A more complicated model was considered in
\cite{Vaillancourt00,Vaillancourt02} by including temperature and vapor mixing
ratio field. The authors investigated the influence of nonuniformity in the
spatial distribution of sizes and positions of cloud droplets on the droplet
size distribution. The relationship among preferential concentration,
sedimentation and Stokes number were also discussed.  Similar to
\cite{Vaillancourt02} and \cite{Lanotte2009}, \cite{Kumar11,Kumar12} developed
a particle resolved DNS to study entrainment mixing processes. In their work, a
slab-like vapor field is adopted to mimic the supersaturated cloudy area and
subsaturated environment. The effects of temperature and buoyancy are ignored
while an artificial isotropic volume forcing is introduced to maintain the
turbulence. In \cite{Kumar14}, the authors extended their previous work to both
forced and decaying turbulence, and claimed that the buoyancy due to droplet
evaporation played minor role in the mixing process.

Furthermore, most DNS models have been based on pseudo-spectral methods due to
its superior accuracy \cite{Rogallo81,Orszag72}. In recent years,
pseudo-spectral method has been extensively applied to the numerical study of
cloud entrainment problem in turbulent flows \cite{And04,Celani05,Kumar11}.
However, the standard pseudo-spectral method has some limitations. As claimed
in \cite{Kumar11}, the spectral method requires a smooth initial condition to
avoid the Gibbs phenomenon. However, a very sharp or zeroth-order discontinuous
interface is reported to exist in the cloud structure on the smallest
observable scales \cite{Brenguier1993}. Therefore, this artificial initial
condition is not realistic or physical and has potential to bring some
imperceptible errors during the simulation. Moreover, the spectral method
requires periodic boundary condition in each direction and cannot be applied to
flows that require a non-periodic, physical boundary condition. To be flexible
enough to deal with various initial profiles with sharp cloud-air interfaces as
well as applying different boundary conditions in the future, we develop a new
particle-resolved DNS (directed numerical simulation) 
using finite difference method coupled with WENO (wighted essentially non-oscillatory) 
scheme \cite{WENO96}, which has the capability of
dealing with discontinuity without causing numerical overshoots at sharp
interfaces.

In the numerical point of view, it's worthwhile to notice that 
the numerical model in this research is similar with the one in parachute simulation 
although they are arising from different fields. 
In the DNS of entrainment-mixing process, the cloud droplets are represented as a group of independent particles, and in the parachute simulation, the parachute canopy is built with a connected particle system. Both of the models can be abstractly described by particle system 
dynamics coupling with fluid dynamics, and therefore the numerical implementation of these 
models should be similar. However, there still several noteworthy difference between these two problems. Firstly, the cloud droplets have no direct interactions with 
each other during entrainment-mixing process, while in parachute simulation a spring mass point are connected with its neighbors and respond to their forces. This requires to add some information in the particle data structure to maintain the connection relationship. Secondly, the entrainment-mixing processes is actually a two-phase flow with phase transition between liquid water and water vapor. Thus a set of dynamic equations describing the water exchange is needed. In addition, the collision handling of these two problems are slightly different. Since the spring mass points are used to construct the fabric surface, 
not only the point-point collision but also point-face and edge-edge collision should be considered to prevent the fabric-self intersections. In conclusion, particles are objects with no extent, while having mass, position and velocity, and responding to forces. Because they are simple, particles can be regarded as the easiest objects to simulate. In spite of simplicity, particles can be powerful to model a wide range of interesting behavior.