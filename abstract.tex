A computational framework that combines the Eulerian description of the turbulence field with a Lagrangian particles ensemble has been proposed in this dissertation. The turbulent field is simulated using direct numerical simulation or eddy viscosity model depending on the turbulence Reynolds number. In the meanwhile, the associated scalar field, such as temperature, mixing ratio, turbulence kinetic energy and dissipation rate, are solved using a second order algorithm combing Crank-Nicholson scheme for time discretization and WENO scheme for the evaluation of hyperbolic term. Moreover, the particle dynamics, such as spring-mass system and cloud droplets, are modeled using ordinary differential system, which are stiff, and hence poses challenge to the stability of the entire system. We have applied this work to the numerical study of parachute deceleration and cloud microphysics. These two distinct problems can be uniformly modeled with PDEs and ODEs, and numerically solved in the same framework. For the parachute simulation, a novel porosity model is proposed to simulate the porous effects of the parachute canopy. This model is easy to implement with the projection method \cite{Brown2001Accurate} and is able to reproduce the Darcy's law observed in the experiment. Furthermore, the impacts of using different versions of $k$-$\epsilon$ model in the parachute simulation has been investigated, and concludes that the standard and RNG model may overestimate the turbulence effects when Reynolds number is small while the Realizable model has a consistent performance with both large and small Reynolds number. In addition, a robust collision resolving algorithm has been developed to enhance the parachute folding procedure. For another application, cloud microphysics, the cloud entrainment and mixing problem is studied in the same numerical framework. The numerical result suggests a new way to parameterize the cloud mixing degree using the dynamical measures. The numerical experiments also verify the negative relationship between the droplets number concentration and the vorticity field. The results imply that the gravity has less impacts on the forced turbulence than the decaying turbulence.

{\it Key Words:} computational fluid dynamics, fluid-structure interactions, particle system dynamics

